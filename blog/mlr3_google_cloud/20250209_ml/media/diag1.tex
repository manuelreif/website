\documentclass[tikz,border=10pt]{standalone}
\usetikzlibrary{positioning,fit}
\begin{document}
\begin{tikzpicture}[node distance=1.2cm, auto,
    box/.style={draw, rectangle, rounded corners, align=center, minimum width=2.5cm, minimum height=1cm, font=\small},
    diamond/.style={draw, diamond, aspect=2, align=center, inner sep=2pt, font=\small},
    ellipse/.style={draw, ellipse, align=center, minimum width=3cm, minimum height=1cm, font=\small},
    arrow/.style={->, thick}]

% --- Knoten ---

% Obere Stufe
\node[box, fill=lightblue] (robustify) {Robustify};

\node[diamond, fill=lightgreen, below=of robustify] (branch) {gabelung\\(unknown\_as\_category / impute\_unknown)};

% Linker Zweig
\node[box, fill=orange, below left=of branch, xshift=-1cm] (imp_d_imp) {Impute (numeric)\\imp\_d};

% Rechter Zweig
\node[box, fill=orange, below right=of branch, xshift=1cm] (colapply) {Colapply\\unknown\_as\_na};
\node[box, fill=orange, below=of colapply] (imp_f_uk) {Impute (factor)\\imp\_f\_uk};
\node[box, fill=orange, below=of imp_f_uk] (imp_d_uk) {Impute (numeric)\\imp\_d\_uk};

% Zusammenführung: Unbranch
% Wir nutzen |- Pfeile, sodass der letzte Streckenabschnitt vertikal erfolgt.
\node[box, fill=plum, below=of imp_d_uk, yshift=-0.5cm, minimum width=3cm] (unbranch) {Unbranch\\unbranch\_main};

% --- Basis-Learner als separater Cluster ---
% Die Basis-Learner werden horizontal angeordnet.
\node[box, fill=yellow, below=of unbranch, xshift=-4cm] (nop) {nop\_original\\(Original)};
\node[box, fill=yellow, right=of nop] (rf) {RF\\(cvrf)};
\node[box, fill=yellow, right=of rf] (nb) {NB\\(cvnb)};
\node[box, fill=yellow, right=of nb] (knn) {kNN\\(cvknn)};

% Cluster um die Basis-Learner (dashed Rahmen)
\node[draw, dashed, inner sep=5pt, fit=(nop) (rf) (nb) (knn), label=above:{Base Learners}] (cluster_base) {};

% Feature Union
\node[ellipse, fill=gray, below=of rf, yshift=-1cm] (base_union) {Feature Union\\base\_learners};

% Encode und finaler Learner
\node[box, fill=lightblue, below=of base_union] (encode) {Encode\\(One-Hot)\\encode\_super};
\node[box, fill=red, below=of encode] (xgboost) {Learner\\XGBoost\\super};

% --- Kanten (Pfeile) ---

\draw[arrow] (robustify) -- (branch);

\draw[arrow] (branch.south) -- (imp_d_imp.north);
\draw[arrow] (branch.south) -- (colapply.north);

\draw[arrow] (colapply.south) -- (imp_f_uk.north);
\draw[arrow] (imp_f_uk.south) -- (imp_d_uk.north);

% Pfeile von den Imputationsergebnissen zum Unbranch-Knoten:
\draw[arrow] (imp_d_imp.south) |- (unbranch.west);
\draw[arrow] (imp_d_uk.south)  |- (unbranch.east);

% Verteilung von unbranch an alle Basis-Learner:
\draw[arrow] (unbranch.south) -- (nop.north);
\draw[arrow] (unbranch.south) -- (rf.north);
\draw[arrow] (unbranch.south) -- (nb.north);
\draw[arrow] (unbranch.south) -- (knn.north);

% Pfeile von den Basis-Learnern in den Feature Union-Knoten:
\draw[arrow] (nop.south) -- (base_union.north);
\draw[arrow] (rf.south)  -- (base_union.north);
\draw[arrow] (nb.south)  -- (base_union.north);
\draw[arrow] (knn.south) -- (base_union.north);

\draw[arrow] (base_union.south) -- (encode.north);
\draw[arrow] (encode.south) -- (xgboost.north);

\end{tikzpicture}
\end{document}
